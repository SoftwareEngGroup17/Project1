%%%%%%%%%%%%%%%%%%%%%%%%%%%%%%%%%%%%%%%%%%%%%%%%%%%%%%%%%%%%%%%%%%%%%%%%%%%%%%%%
%                                                                              %
%                     ELEN4009 - Software Engineering                          %
%                                                                              %
%                         Sprint Retrospective                                 %
%                                 for                                          %
%         The Online Postgraduate Application Approval System for EIE          %
%                                                                              %
%             Julio Baeta (710066), Nomakhosi Ndebele (671480),                %
%             Ryan Robinson (453764),Timothy Rokebrand (458960)                %
%                                                                              %
%          This LaTeX document was adapted from the IEEE template,             %
%                           which can be found at:                             %
% http://www.ieee.org/conferences_events/conferences/publishing/templates.html %
%                                                                              %
%%%%%%%%%%%%%%%%%%%%%%%%%%%%%%%%%%%%%%%%%%%%%%%%%%%%%%%%%%%%%%%%%%%%%%%%%%%%%%%%

\documentclass[journal,comsoc,onecolumn]{IEEEtran}
\usepackage[T1]{fontenc}
\usepackage{fancyhdr}

%%%%%%%%%%%%%%%%%%%%%%%%%%%%%%%%%%%%%%%%%%%%%%%%%%%%%%%%%%%%%%%%%%%%%%%%%%%%%%%%

\begin{document}

%%%%%%%%%%%%%%%%%%%%%%%%%%%%%%%%%%%%%%%%%%%%%%%%%%%%%%%%%%%%%%%%%%%%%%%%%%%%%%%%

\title{Sprint Retrospective \\ \vspace{7mm} for \\ \vspace{7mm} The Online Postgraduate Application \\ Approval System for EIE}

\author{\vspace{3mm} Julio Baeta (710066), Nomakhosi Ndebele (671480), Ryan Robinson (453764), Timothy Rokebrand (458960)\\ \small \vspace{2mm} School of Electrical \& Information Engineering, University of the Witwatersrand, Private Bag 3, 2050, Johannesburg, South Africa}

\markboth{}{}

\maketitle

\thispagestyle{empty}

%%%%%%%%%%%%%%%%%%%%%%%%%%%%%%%%%%%%%%%%%%%%%%%%%%%%%%%%%%%%%%%%%%%%%%%%%%%%%%%%

\newpage

\thispagestyle{empty}

\section{SPRINT RETROSPECTIVE}

%%%%%%%%%%%%%%%%%%%%%%%%%%%%%%%%%%%%%%%%%%%%%%%%%%%%%%%%%%%%%%%%%%%%%%%%%%%%%%%%

The following document is based on lecture slides provided by Professor Otoo for the ELEN4009 course at the University of the Witwatersrand (2016). \newline

In order to ensure that each successive sprint was as productive as possible, every sprint needed to be assessed and new practices adopted for the following sprint. This process allowed for the team to determine which aspects of each sprint enabled them to be as productive as they possibly could.

%%%%%%%%%%%%%%%%%%%%%%%%%%%%%%%%%%%%%%%%%%%%%%%%%%%%%%%%%%%%%%%%%%%%%%%%%%%%%%%%

\subsection{19 February - Sprint Planning Meeting}
This was the first sprint meeting and as a result, there was no previous sprint on which the team could gather data from.

%%%%%%%%%%%%%%%%%%%%%%%%%%%%%%%%%%%%%%%%%%%%%%%%%%%%%%%%%%%%%%%%%%%%%%%%%%%%%%%%

\subsection{26 February - Sprint Planning Meeting}

\subsubsection{Successful aspects of previous sprint}
\hfill \break Team learned that a website development application (such as SiteBuilder.com or Sitey.com)  would not be acceptable for the scope of the project. Realised that software such as HTML5 and CSS would need to be used.

\subsubsection{Weak points of previous sprint}
\hfill \break A significant amount of time was wasted looking into various website development applications that could not be used.

\subsubsection{Improvements for next sprint}
\hfill \break Read the course brief and outline for the ELEN4009 course, since examples of the types of software that could be used for the project were stated within this document.

%%%%%%%%%%%%%%%%%%%%%%%%%%%%%%%%%%%%%%%%%%%%%%%%%%%%%%%%%%%%%%%%%%%%%%%%%%%%%%%%

\subsection{4 March - Sprint Planning Meeting}

\subsubsection{Successful aspects of previous sprint}
\hfill \break Managed to decide on what software should be utilised within the project.

\subsubsection{Weak points of previous sprint}
\hfill \break Team was not completely proficient in the software chosen for both the back and front ends of the system.

\subsubsection{Improvements for next sprint}
\hfill \break Allow for more time within the upcoming sprint for learning how to make use of the software in the correct manner.

%%%%%%%%%%%%%%%%%%%%%%%%%%%%%%%%%%%%%%%%%%%%%%%%%%%%%%%%%%%%%%%%%%%%%%%%%%%%%%%%

\subsection{11 March - Sprint Planning Meeting}

\subsubsection{Successful aspects of previous sprint}
\hfill \break Development team learned exactly how the current system in place works and what they would need to design in order to be successful.

\subsubsection{Weak points of previous sprint}
\hfill \break Did not consult with the product owner. As a result, scope changed drastically after meeting with the Postgraduate Officer.

\subsubsection{Improvements for next sprint}
\hfill \break Need to work more closely with the product owner, since they define what the goals of the software are.

%%%%%%%%%%%%%%%%%%%%%%%%%%%%%%%%%%%%%%%%%%%%%%%%%%%%%%%%%%%%%%%%%%%%%%%%%%%%%%%%

\subsection{18 March - Sprint Planning Meeting}

\subsubsection{Successful aspects of previous sprint}
\hfill \break Team formed a good idea of how the web pages would be structured and how they would appear to the user.

\subsubsection{Weak points of previous sprint}
\hfill \break Database management system (DBMS) selected (MySQL) took longer than expected to set up and configure, which took away time from the text database setup and configuration (Solr).

\subsubsection{Improvements for next sprint}
\hfill \break Consult more sources and tutorials as to how the software being used should be configured and set up.

%%%%%%%%%%%%%%%%%%%%%%%%%%%%%%%%%%%%%%%%%%%%%%%%%%%%%%%%%%%%%%%%%%%%%%%%%%%%%%%%

\subsection{29 March - Sprint Planning Meeting}

\subsubsection{Successful aspects of previous sprint}
\hfill \break MySQL was correctly configured and set up.
Interim tests performed on the database showed promising results.
The various web page views began to link with one another and the styling began to look aesthetically pleasing.

\subsubsection{Weak points of previous sprint}
\hfill \break Table creation, population and integration (in HTML5) with the MySQL database proved to be challenging and time-consuming. Styling and link creation within HTML also proved to be a challenge.
Initial test database that was constructed did not contain enough fields for the information required.
\break Solr had not yet been constructed and tested, since more time was being spent on MySQL than initially planned.

\subsubsection{Improvements for next sprint}
\hfill \break More effort had to be put into ensuring that the linking between web pages was correct and that the system was easy to use.
More time had to be spent on implementing the text database system (Solr) by the back-end team.

%%%%%%%%%%%%%%%%%%%%%%%%%%%%%%%%%%%%%%%%%%%%%%%%%%%%%%%%%%%%%%%%%%%%%%%%%%%%%%%%

\subsection{5 April - Sprint Planning Meeting}

\subsubsection{Successful aspects of previous sprint}
\hfill \break Team was finally able to observe how the system would function as a whole when it was finally completed.
It was discovered that the most productive manner of working was to keep lines of communication between the front-end and back-end groups consistently open.

\subsubsection{Weak points of previous sprint}
\hfill \break Integration between the front and back ends proved to be somewhat of a challenge for the software development team. This is due to each team having become proficient in their own specific software modules, but weren't sure how it linked up with the other module.

\subsubsection{Improvements for next sprint}
\hfill \break The team needed to ensure, that for the final days leading up to the submission deadline, communication within the entire team was as effective as possible.

%%%%%%%%%%%%%%%%%%%%%%%%%%%%%%%%%%%%%%%%%%%%%%%%%%%%%%%%%%%%%%%%%%%%%%%%%%%%%%%%

\subsection{7 April - Sprint Planning Meeting}

\subsubsection{Successful aspects of previous sprint}
\hfill \break Due to the constant evaluation of sprints, the final sprint was found to be highly productive. This then led to a large quantity of work being completed within the final sprint.

\subsubsection{Weak points of previous sprint}
\hfill \break The team discovered that it would have been more effective and time-efficient to write all of the relevant documentation required whilst the software was being implemented.
\break It was also established that the set of skills possessed by an additional Information Engineering (IE) student within the group would have  been highly beneficial to the project.

\subsubsection{Improvements for next sprint}
\hfill \break The final product was submitted on this day, and subsequently no further sprint planning meetings would be taking place.

%%%%%%%%%%%%%%%%%%%%%%%%%%%%%%%%%%%%%%%%%%%%%%%%%%%%%%%%%%%%%%%%%%%%%%%%%%%%%%%%

\end{document}

%%%%%%%%%%%%%%%%%%%%%%%%%%%%%%%%%%%%%%%%%%%%%%%%%%%%%%%%%%%%%%%%%%%%%%%%%%%%%%%%