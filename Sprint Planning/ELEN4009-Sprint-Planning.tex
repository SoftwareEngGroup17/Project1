%%%%%%%%%%%%%%%%%%%%%%%%%%%%%%%%%%%%%%%%%%%%%%%%%%%%%%%%%%%%%%%%%%%%%%%%%%%%%%%%
%                                                                              %
%                     ELEN4009 - Software Engineering                          %
%                                                                              %
%                            Sprint Planning                                   %
%                                 for                                          %
%         The Online Postgraduate Application Approval System for EIE          %
%                                                                              %
%             Julio Baeta (710066), Nomakhosi Ndebele (671480),                %
%             Ryan Robinson (453764),Timothy Rokebrand (458960)                %
%                                                                              %
%          This LaTeX document was adapted from the IEEE template,             %
%                           which can be found at:                             %
% http://www.ieee.org/conferences_events/conferences/publishing/templates.html %
%                                                                              %
%%%%%%%%%%%%%%%%%%%%%%%%%%%%%%%%%%%%%%%%%%%%%%%%%%%%%%%%%%%%%%%%%%%%%%%%%%%%%%%%

\documentclass[journal,comsoc,onecolumn]{IEEEtran}
\usepackage[T1]{fontenc}
\usepackage{fancyhdr}

%%%%%%%%%%%%%%%%%%%%%%%%%%%%%%%%%%%%%%%%%%%%%%%%%%%%%%%%%%%%%%%%%%%%%%%%%%%%%%%%

\begin{document}

%%%%%%%%%%%%%%%%%%%%%%%%%%%%%%%%%%%%%%%%%%%%%%%%%%%%%%%%%%%%%%%%%%%%%%%%%%%%%%%%

\title{Sprint Planning \\ \vspace{7mm} for \\ \vspace{7mm} The Online Postgraduate Application \\ Approval System for EIE}

\author{\vspace{3mm} Julio Baeta (710066), Nomakhosi Ndebele (671480), Ryan Robinson (453764), Timothy Rokebrand (458960)\\ \small \vspace{2mm} School of Electrical \& Information Engineering, University of the Witwatersrand, Private Bag 3, 2050, Johannesburg, South Africa}

\markboth{}{}

\maketitle

\thispagestyle{empty}

%%%%%%%%%%%%%%%%%%%%%%%%%%%%%%%%%%%%%%%%%%%%%%%%%%%%%%%%%%%%%%%%%%%%%%%%%%%%%%%%

\newpage

\thispagestyle{empty}

\section{SPRINT PLANNING}

%%%%%%%%%%%%%%%%%%%%%%%%%%%%%%%%%%%%%%%%%%%%%%%%%%%%%%%%%%%%%%%%%%%%%%%%%%%%%%%%

The following document is based on content found within the textbook "Beginning Software Engineering" by Rod Stephens (pages 327 - 329) as well as lecture slides provided by Professor Otoo for the ELEN4009 course at the University of the Witwatersrand (2016). \newline

The adopted method for the software development life cycle (SDLC) was that of SCRUM. In order to ensure that this methodology was carried out in the correct way, a number of details within the project had to be specified.

%%%%%%%%%%%%%%%%%%%%%%%%%%%%%%%%%%%%%%%%%%%%%%%%%%%%%%%%%%%%%%%%%%%%%%%%%%%%%%%%

\subsection{Scrum Roles}

\begin{itemize}
	\item Product Owner: The customer for this product will be the School of Electrical and Information Engineering within the University of the Witwatersrand.
	\item Users: The users of the product will be the Postgraduate Officer (Ms Mumtaz Adam), Posgraduate Programme Coordinator (Prof. Ivan Hofsajer), the relevant supervisors for a particular applicant's field of interest as well as the applicants themselves.
	\item Other Stakeholders: Student Information and Management Systems (SIMS), which is the department at the university that deals with all information pertaining to an applicant as well as the Student Enrolment Centre (SEnC), that deals with the initial application of the potential students.
\end{itemize}

\subsection{Specific Team Roles}

\begin{itemize}
	\item Team members: The team members refers to the software development team that will be working on the various modules of the project. This will include Julio Baeta, Nomakhosi Ndebele, Ryan Robinson and Timothy Rokebrand.
	\item Scrum Master: One of the team members was elected to fulfil the role of a leader, which is more commonly known as a 'scrum master' within the context of software engineering projects. This role required the scheduling of meetings, coordinating of group members and clarifying unclear aspects of the project with the course lecturer. Within the scope of this project, the scrum master also acted as one of the team members. Ryan Robinson fulfilled this role for the group.
\end{itemize}

%%%%%%%%%%%%%%%%%%%%%%%%%%%%%%%%%%%%%%%%%%%%%%%%%%%%%%%%%%%%%%%%%%%%%%%%%%%%%%%%

\subsection{Scrum Sprints}

\hfill \break Due to the nature of the project and the limited scope that would be able to be implemented, sprint planning meetings took place once a week and hence each individual sprint lasted one week. Usually, sprint lengths are between two to four weeks, but due to the nature of the project, this duration had to be reduced in order to ensure that the project was finished on time.

\hfill \break Sprint planning meetings took place every Friday at 10am and concluded either at 12pm or 1pm, thus ensuring that an unnecessarily long amount of time. Generally, the product owner will be present at each sprint planning meeting, however, for this project, the owner (school of EIE) was very specific from the beginning about what they required in terms of functionality and the software's requirements. As a result of this, only one meeting with the product owner was held and a product backlog produced from that meeting. Additional aspects of the project were then obtained through course laboratory exercises, as well as being deduced by the group.

\hfill \break Another aspect of SCRUM is that of a daily scrum meeting.However, due to the software development team members having different commitments and responsibilities, it was not possible for daily scrum meetings to be held. Therefore, the project relied solely on weekly sprint planning meetings.

\hfill \break Usually within industry, a functional module is produced at the end of each sprint. Although some form of software was produced at the end of most sprints, the module produced was not always fully functional. This can be attributed to the lack of experience in the languages and software used by the software developers working on the project.

%%%%%%%%%%%%%%%%%%%%%%%%%%%%%%%%%%%%%%%%%%%%%%%%%%%%%%%%%%%%%%%%%%%%%%%%%%%%%%%%

\subsection{19 February}

\begin{itemize}
	\item Front End Goals for the Upcoming Sprint: Decide on what software should be used for the project.
	\item Plan to achieve Front End Goals for the Upcoming Sprint: Research the possible various software modules that can be used.
	\item Back End Goals for the Upcoming Sprint: Decide on what software should be used for the project.
	\item Plan to achieve Back End Goals for the Upcoming Sprint: Research the possible various software modules that can be used.
\end{itemize}

%%%%%%%%%%%%%%%%%%%%%%%%%%%%%%%%%%%%%%%%%%%%%%%%%%%%%%%%%%%%%%%%%%%%%%%%%%%%%%%%

\subsection{26 February}

\begin{itemize}
	\item Front End Goals for the Upcoming Sprint: Look into using web page development software.
	\item Plan to achieve Front End Goals for the Upcoming Sprint: Research and learn how to use Wix, CSS and HTML.
	\item Back End Goals for the Upcoming Sprint: Look into DBMS software modules (database and text database).
	\item Plan to achieve Back End Goals for the Upcoming Sprint: Research and learn how to use SQL and Solr.
\end{itemize}

%%%%%%%%%%%%%%%%%%%%%%%%%%%%%%%%%%%%%%%%%%%%%%%%%%%%%%%%%%%%%%%%%%%%%%%%%%%%%%%%

\subsection{4 March}

\begin{itemize}
	\item Front and Back End Goals for the Upcoming Sprint: Speak to Postgraduate Officer in order to establish the customer’s needs
	Write up the Software Requirements Specification (SRS).
	\item Plan to achieve Front and Back End Goals for the Upcoming Sprint: Speak to the School of Electrical and Information Engineering administration and email the Postgraduate Officer (Ms Adam) to set up an appointment
\end{itemize}

%%%%%%%%%%%%%%%%%%%%%%%%%%%%%%%%%%%%%%%%%%%%%%%%%%%%%%%%%%%%%%%%%%%%%%%%%%%%%%%%

\subsection{11 March}

\begin{itemize}
	\item Front End Goals for the Upcoming Sprint: Design of web page layouts and how the navigation between them will occur.
	\item Plan to achieve Front End Goals for the Upcoming Sprint: Plan and discuss using diagrams drawn on paper, divide tasks into HTML and CSS.
	\item Back End Goals for the Upcoming Sprint: Setting up and configuring of MySQL.
	\item Plan to achieve Back End Goals for the Upcoming Sprint: Look at phpMyAdmin.
\end{itemize}

%%%%%%%%%%%%%%%%%%%%%%%%%%%%%%%%%%%%%%%%%%%%%%%%%%%%%%%%%%%%%%%%%%%%%%%%%%%%%%%%

\subsection{18 March}

\begin{itemize}
	\item Front End Goals for the Upcoming Sprint: Start to implement the design into HTML and CSS.
	\item Plan to achieve Front End Goals for the Upcoming Sprint: Design each page separately and subsequently link them up.
	\item Back End Goals for the Upcoming Sprint: Create a test database and experiment with its functionality.
	\item Plan to achieve Back End Goals for the Upcoming Sprint: Set up Apache, PHP and MySQL.
\end{itemize}

%%%%%%%%%%%%%%%%%%%%%%%%%%%%%%%%%%%%%%%%%%%%%%%%%%%%%%%%%%%%%%%%%%%%%%%%%%%%%%%%

\subsection{29 March}

\begin{itemize}
	\item Front and Back End Goals for the Upcoming Sprint: Start integration process between the database and the front end.
	\item Plan to achieve Front and Back End Goals for the Upcoming Sprint: Teams will convene and explain the inner workings of the separate ends to one another.
	\item Back End Goals for the Upcoming Sprint: Setting up and configuring of Solr.
	\item Plan to achieve Back End Goals for the Upcoming Sprint: Install Solr, along with a database connector and configure it.
\end{itemize}

%%%%%%%%%%%%%%%%%%%%%%%%%%%%%%%%%%%%%%%%%%%%%%%%%%%%%%%%%%%%%%%%%%%%%%%%%%%%%%%%

\subsection{5 April}

\begin{itemize}
	\item Front and Back End Goals for the Upcoming Sprint: Finalisation of the front and back end modules. Writing up of all remaining documentation and editing of the documents already written and testing of the final system.
	\item Plan to achieve Front and Back End Goals for the Upcoming Sprint: Check that all requirements had been met (ones set by the team as well as the ones required for the project, discuss what should appear within each document and test each individual module in isolation as well as the overall module. 
\end{itemize}

%%%%%%%%%%%%%%%%%%%%%%%%%%%%%%%%%%%%%%%%%%%%%%%%%%%%%%%%%%%%%%%%%%%%%%%%%%%%%%%%

\end{document}

%%%%%%%%%%%%%%%%%%%%%%%%%%%%%%%%%%%%%%%%%%%%%%%%%%%%%%%%%%%%%%%%%%%%%%%%%%%%%%%%