%%%%%%%%%%%%%%%%%%%%%%%%%%%%%%%%%%%%%%%%%%%%%%%%%%%%%%%%%%%%%%%%%%%%%%%%%%%%%%%%
%                                                                              %
%                     ELEN4009 - Software Engineering                          %
%                                                                              %
%                        Laboratory Exercise No 2                              %
%                                                                              %
%             Julio Baeta (710066), Nomakhosi Ndebele (671480),                %
%             Ryan Robinson (453764),Timothy Rokebrand (458960)                %
%                                                                              %
%          This LaTeX document was adapted from the IEEE template,             %
%                           which can be found at:                             %  % http://www.ieee.org/conferences_events/conferences/publishing/templates.html %
%                                                                              %
%%%%%%%%%%%%%%%%%%%%%%%%%%%%%%%%%%%%%%%%%%%%%%%%%%%%%%%%%%%%%%%%%%%%%%%%%%%%%%%%

\documentclass[journal,comsoc]{IEEEtran}

\usepackage[T1]{fontenc}

%%%%%%%%%%%%%%%%%%%%%%%%%%%%%%%%%%%%%%%%%%%%%%%%%%%%%%%%%%%%%%%%%%%%%%%%%%%%%%%%

\begin{document}

\title{ELEN4009 - SOFTWARE ENGINEERING\\ LABORATORY EXERCISE 2}

\author{\vspace{3mm} Julio Baeta (710066), Nomakhosi Ndebele (671480), Ryan Robinson (453764), Timothy Rokebrand (458960)\\ \small \vspace{2mm} School of Electrical \& Information Engineering, University of the Witwatersrand, Private Bag 3, 2050, Johannesburg, South Africa
}

% The paper headers
\markboth{}{}

\maketitle

\thispagestyle{empty}
\pagestyle{empty}

\IEEEpeerreviewmaketitle

%%%%%%%%%%%%%%%%%%%%%%%%%%%%%%%%%%%%%%%%%%%%%%%%%%%%%%%%%%%%%%%%%%%%%%%%%%%%%%%%

\section{THE USER INTERFACE - FRONT END}

\hfill \break The complete application is received, from SIMS, by the School's page. The application is first assessed by the postgraduate officer (Ms. Adam) to ensure that all the information is present. If not, a message is sent back to SIMS to inform them that the application is incomplete. When all the information is present the postgraduate officer gives the relevant supervisors access to the applicants details and proposal. It is now up to the supervisor to make a decision on whether the applicant has been successful. If the application is rejected reasons must be given. Once the supervisor has made his decision, the head supervisor (Prof. Hofsajer) is able to view the decision and either finalise the decision or query it with the relevant supervisor. \break

%%%%%%%%%%%%%%%%%%%%%%%%%%%%%%%%%%%%%%%%%%%%%%%%%%%%%%%%%%%%%%%%%%%%%%%%%%%%%%%%

\subsubsection{Applying Student Interface (Timothy Rokebrand)}

\hfill \break HTML This module will consist of coding, the main responsibility to take the proposed design and ensure that the functionality is as expected (i.e. the buttons do what is expected, hyperlinks navigate to the correct pages etc.), linking the html code with CSS and assisting with the linking of the front-end to the back-end
\vspace{2mm}

%%%%%%%%%%%%%%%%%%%%%%%%%%%%%%%%%%%%%%%%%%%%%%%%%%%%%%%%%%%%%%%%%%%%%%%%%%%%%%%%

\subsubsection{Administrative Officer Interface (Nomakhosi Ndebele)}

\hfill \break This aspect will be broken down into: CSS coding, the main responsibility is the design and layout of the page i.e. what everything will look like and what needs to be on each page, as well as linking the html code with CSS.

%%%%%%%%%%%%%%%%%%%%%%%%%%%%%%%%%%%%%%%%%%%%%%%%%%%%%%%%%%%%%%%%%%%%%%%%%%%%%%%%

\section{DATABASE - THE BACK END}

%%%%%%%%%%%%%%%%%%%%%%%%%%%%%%%%%%%%%%%%%%%%%%%%%%%%%%%%%%%%%%%%%%%%%%%%%%%%%%%%

\subsubsection{Student Information Database (Ryan Robinson)}

\hfill \break This database will be used to store all of the information pertaining to the student such as their name, student number, academic history and contact details. For the course of this project, the assumption will be made that all of the applying student's information is already present on the university's database. This implies that the student's information has already been entered by the Wits Students Information Management System (SIMS) once an application has been received by the School of Electrical and Information Engineering. \break

The database management system that will be used for the student's information will be MySQL. This database management system will also be used in conjunction with Apache webserver software. Since the webpage (user) interface or the 'front-end' of the system will be making use of HyperText Markup Language (HTML), the scripting language Hypertext Preprocessor (PHP) will be used as a means of creating a link between the webpage and the relevant databases.
\vspace{2mm}

%%%%%%%%%%%%%%%%%%%%%%%%%%%%%%%%%%%%%%%%%%%%%%%%%%%%%%%%%%%%%%%%%%%%%%%%%%%%%%%%

\subsubsection{Document Storage Database (Julio Baeta)}

This database will be used to store any relevant documents that are required to be submitted by the student as part of their postgraduate application process. This includes a curriculum vitae (CV), a copy of their degree as well as a copy of academic records if they are applying from a different university (did not complete their undergraduate degree at the University of the Witwatersrand). The assumption that all documentation has already been entered into the university's database by SIMS will also be made for this portion of the database aspect. The database management system that will be used for the student's documentation will be Solr. Alternatively, MySQL could be used for this as well (by making use of Binary Large Objects [BLOBs]), but this has been known to cause what is known as 'table bloat' (where MySQL uses too much memory and causes performance issues) in addition to causing a variety of other problems [1].

\break A Solr and Lucene combination was selected since it is open source, has a large communtity and are packaged together as one product by Apache. Solr also provides an advanced full text search, which will allow postgraduate officers to find the proposal though key words instead of teadiously searching though the database. Solr uses XML, JSON and HTTP. Apache Zookeeper makes the database easily scalable, this combined with the open source nature should greatly extend the products life and allow it to quickly change according to the client's needs. Solr also allows the use of security SSL/TLS which is a needed requirement for connecting to the internet today [2]. Thus the most optimal backend will use MySQL to create the basic database which will be used in Solr/Lucene with its extended features.
\hfill \break

%%%%%%%%%%%%%%%%%%%%%%%%%%%%%%%%%%%%%%%%%%%%%%%%%%%%%%%%%%%%%%%%%%%%%%%%%%%%%%%%

%\section{Conclusion}
%The conclusion goes here.

%%%%%%%%%%%%%%%%%%%%%%%%%%%%%%%%%%%%%%%%%%%%%%%%%%%%%%%%%%%%%%%%%%%%%%%%%%%%%%%%

\section{REFERENCES}

[1] Voelkel, J. (2014). Performance: WP-Options Table Bloat and It’s Performance Effects. Available from: http://justinvoelkel.me/problem-solved- wp-options- table-bloat- and-its- performance-effects/. Last accessed 7th March 2016. \hfill \break

[2] The Apache Software Foundation. Solr Features. Avaliable from: http://lucene.apache.org/solr/features.html. Last accessed 6th March 2016. \hfill \break

%%%%%%%%%%%%%%%%%%%%%%%%%%%%%%%%%%%%%%%%%%%%%%%%%%%%%%%%%%%%%%%%%%%%%%%%%%%%%%%%

\end{document}

%%%%%%%%%%%%%%%%%%%%%%%%%%%%%%%%%%%%%%%%%%%%%%%%%%%%%%%%%%%%%%%%%%%%%%%%%%%%%%%%